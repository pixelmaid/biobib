\documentclass[10pt]{article}
\usepackage[bottom=1in,left=1in,top=1in,right=0.75in]{geometry}
\usepackage{tabularx}
\usepackage{fancyhdr}
\usepackage{supertabular}
\usepackage{longtable}
\usepackage{makecell}
\usepackage{array}
\usepackage[linkcolor=red,urlcolor=red]{hyperref}
\usepackage[official]{eurosym}

\setcounter{LTchunksize}{5}

\pagestyle{fancy}
\lhead{(Page \thepage)}
\cfoot{}
\renewcommand{\headrulewidth}{0pt}


\begin{document}

\noindent
\begin{tabularx}{\linewidth}{XX} 
BIO-BIBLIOGRAPHY & \hfill University of California, Santa Barbara \\ \\
 & \hfill September 15th, 2020 \\ \\
 Jennifer M. Jacobs &   \\ 
 Assistant Professor II & \\
 Department of Media Arts and Technology & \\
 Director of the Expressive Computation Lab & 
 \end{tabularx}

\vspace{1cm}
% Last update filed on October 6, 2018

This update refers to the period since appointment %July 1, 2016 
to September 15th, 2020


\vspace{1cm}
\centering {\textbf{ \large Curriculum Vitae}}

\raggedright

\vspace{0.5cm}
\underline{Education}

University of Oregon, B.F.A. (honors), Digital Arts, 2007

Hunter College, M.F.A., Integrated Media Arts, 2011

Massachusetts Institute of Technology, M.S., Media Arts and Technology, 2013

Massachusetts Institute of Technology, Ph. D., Media Arts and Technology, 2017


\vspace{0.5cm}
\underline{Area of Specialization}

Digital Fabrication

Computational Design

Human-Centered Programming

Human-Computer Interaction

\vspace{0.5cm}
\underline{Previous Academic or Professional Appointments}
\begin{tabular}{l p{5.5in} }

2009 \ \ \ \  & Designer and Developer, Tiltfactor Game Research Lab, New York \\
2009 - 2011\ \ \ \ & Adjunct Instructor, Film and Media Studies, Hunter College \\ 
2011 - 2017 \ \ \ \ & Research Assistant, MIT Media Lab, Massachusetts Institute of Technology \\
2014 - 2015 & Research Intern, Adobe Research, San Francisco\\
2016 & Artist in Residence, Autodesk Pier 9, San Francisco \\
2016 & Research Intern, Dynamic Medium Group, Human Advancement Research Community, Oakland \\
2017 - 2019\ \ \ \ & Postdoctoral Fellow, Brown Institute For Media Innovation, Computer Science Department, Stanford University\\
2019 - \ \ \ \ & Assistant Professor, Media Arts and Technology, University of California, Santa Barbara\\ 

\end{tabular}

\vspace{0.5cm}
\underline{Professional Organizations}

Member, Association for Computing Machinery (ACM) 

\vspace{0.5cm}

\newpage

\textbf{PART I.  RESEARCH}

\vspace{0.2cm}
{\bf Cumulative List of Publications (or Creative Activities):}

%\vspace{0.2cm}
{\setlength{\extrarowheight}{3.5pt}
% UC Bio-bib Publication Table
% Created on 2020-10-05 16:34

\begin{longtable}{lcp{7.75cm}>{\raggedright}p{5.25cm}p{1.75cm}}
\# & Year & Title and Authors & Publisher & Category\\
\hline 
\endhead 
    1 & 2012.0 & {\bf Codeable Objects}, Jacobs J., Buechley, L..  & \emph{ Open Hardware Summit, Open Source Hardware Association } .   & Conference Demonstration\\
    2 & 2013 & {\bf Codeable Objects: Computational Design and Digital Fabrication for Novice Programmers}, Jacobs J., Buechley L..  & \emph{ ACM Conference on Human Factors in Computing Systems (CHI) } . doi:10.1145/2470654.2466211.  & Refereed Article\\
    3 & 2014 & {\bf DressCode: Supporting Youth in Computational Design and Making}, Jacobs J., Resnick M., Buechley L..  & \emph{ Constructionism Conference  } .   & Refereed Article\\
    4 & 2015 & {\bf Hybrid Practice in the Kalahari: Design Collaboration Through Digital Tools and Hunter Gatherer Craft}, Jacobs J., Zoran  A..  & \emph{ ACM Conference on Human Factors in Computing Systems (CHI) } .   & Refereed Article\\
    5 & 2015 & {\bf Hybrid Craft: Showcase of Physical and Digital Integration of Design and Craft Skills}, Zoran A., Valjakk O., Chan  B., Brosh A. , Gordon  R., Friedman, Y., Marshall,  J., Bunnell,  K., Jorgensen, T. , Factum Arte, Hope, S. , Schmitt, P. , Buechley, L. .  & \emph{  } .   & Refereed Article\\
    6 & 2016 & {\bf Supporting creativity, expressiveness and complexity through personal fabrication.
}, Jacobs J..  & \emph{ XRDS: Crossroads, The ACM Magazine for Students } .   & Magazine Article\\
    7 & 2016 & {\bf Negotiating science, technology, culture, and religion: the art and ideas of Laleh Mehran}, Jacobs J..  & \emph{ XRDS: Crossroads, The ACM Magazine for Students } .   & Magazine Article\\
    8 & 2016 & {\bf From prototype to product: deployment strategies in computer science research}, Jacobs J..  & \emph{ XRDS: Crossroads, The ACM Magazine for Students } .   & Magazine Article\\
    9 & 2017 & {\bf Supporting Expressive Procedural Art Creation through Direct Manipulation}, Jacobs J., Gogia S., Měch R., Brandt J. .  & \emph{ ACM Conference on Human Factors in Computing Systems (CHI) } .   & Refereed Article\\
    10 & 2017 & {\bf Technology in defense of democracy}, Jacobs J..  & \emph{  } . doi:10.1145/3055141.  & Magazine Article\\
    11 & 2017 & {\bf Direct and immediate drawing with CNC machines}, Li. J, Jacobs J., Chang M., Hartmann B..  & \emph{ ACM Symposium on Computational Fabrication } .   & Refereed Article\\
    12 & 2017 & {\bf Multidisciplinary systems engineering
}, Jacobs J..  & \emph{ XRDS: Crossroads, The ACM Magazine for Students } .   & Magazine Article\\
    13 & 2017 & {\bf Finding the edge: art and automation
}, Jacobs J..  & \emph{ XRDS: Crossroads, The ACM Magazine for Students } .   & Magazine Article\\
    14 & 2018.0 & {\bf Dynamic Brushes: Extending Manual Drawing Practices with Artist-Centric Programming Tools}, Jacobs J., J. Brandt, R. Měch, M. Resnick.  & \emph{ ACM Conference on Human Factors in Computing Systems (CHI) } .   & Conference Demonstration\\
    15 & 2018 & {\bf Extending Manual Drawing Practices with Artist-Centric Programming Tools}, Jacobs J., J. Brandt, R. Měch, M. Resnick.  & \emph{ ACM Conference on Human Factors in Computing Systems (CHI) } .   & Refereed Article\\
    16 & 2019 & {\bf SK3TCH: Fast, Interactive 3D Printing via Liquid Crystallization}, Hirsch M., Jacobs J., Visell Y..  & \emph{ ACM Symposium on Computational Fabrication } .   & Conference Abstract\\
\\hline
\\hline
   &   & {\bf Since Appointment:} &    &   \\\\
    17 & 2019 & {\bf Editing Self Image}, Fried O., Jacobs J., Finkelstein A., Agrawala M. . \href{http://jenniferjacobs.mat.ucsb.edu/papers/editingselfimage2019.pdf}{[pdf]} & \emph{ Communications of the ACM (CACM) } .   & Refereed Article\\
    18 & 2020 & {\bf Supporting Visual Artists in Programming through Direct Inspection and Control of Program Execution}, Li, J., Brandt J., Měch R., Agrawala M. , Jacobs J.. \href{http://jenniferjacobs.mat.ucsb.edu/papers/ddb.pdf}{[pdf]} & \emph{ ACM Conference on Human Factors in Computing Systems (CHI) } .   & Refereed Article\\
    19 & 2020 & {\bf Learning Remotely, Making Locally: Teaching Digital Fabrication During a Pandemic}, Jacobs J., Peek N.. \href{https://interactions.acm.org/blog/view/learning-remotely-making-locally-remote-digital-fabrication-instruction-dur}{[pdf]} & \emph{ ACM Interactions } .   & Magazine Article\\
\end{longtable}
 % modified to include url to pdf instead of doi for since last review -- argh
}

\vspace{0.2cm}
{\bf Works In Press:}

%\vspace{0.15cm}
{\setlength{\extrarowheight}{3.5pt}
% UC Bio-bib Publication Table
% Created on 2020-10-04 17:24

\begin{longtable}{lcp{7.75cm}>{\raggedright}p{5.25cm}p{1.75cm}}
\# & Year & Title and Authors & Publisher & Category\\
\hline 
\endhead 
\end{longtable}

}

 

\vspace{0.2cm}

{\bf Work Submitted:}

% For some reason, I need this for the list to appear...
% {\tiny :}

 %\vspace{0.15cm}
 {\setlength{\extrarowheight}{3.5pt}
% UC Bio-bib Publication Table
% Created on 2020-10-05 19:18

\begin{longtable}{lcp{7.75cm}>{\raggedright}p{5.25cm}p{1.75cm}}
\# & Year & Title and Authors & Publisher & Category\\
\hline 
\endhead 
    22 & 2020 & {\bf Remote Learners, Home Makers: How Digital Fabrication Was Taught Online During a Pandemic}, Benabdallah G., Bourgault S., Peek N., Jacobs J..  & \emph{ ACM Conference on Human Factors in Computing Systems (CHI) } .   & Refereed Article\\
    23 & 2020 & {\bf What Visual Artists Can Teach Us About Software Development}, Li J., Hashim S., Jacobs J..  & \emph{ ACM Conference on Human Factors in Computing Systems (CHI) } .   & Refereed Article\\
\end{longtable}

}


\vspace{1cm}
\textbf{PART II.  TEACHING (50\% Geography Dept. workload; 50\% Bren School workload. Teaching release throughout the review period.)}

\vspace{0.5cm}

\textbf{Workload Descriptions}

\begin{enumerate}
{\footnotesize

\item {\em Geography Department}: Geography department teaching workload for full-time ladder faculty is 3 Instructional Workload Courses (IWC) per year. 50\% workload is 1.5 IWC per year.

\item {\em Bren School}:  The official teaching workload in the Bren School for full-time ladder faculty members is 3.5 Instructional Workload Courses (IWC) per academic year. Full-time faculty are expected to be in residence during the academic year and teach at least two out of three quarters unless they are on sabbatical or approved leave. Each quarter on sabbatical counts as 1.17 IWC for full-time faculty. The teaching workload for faculty with partial appointments is the equivalent fraction of 3.5 and the teaching workload is negotiated with the dean.

Every full-time faculty member will advise or co-advise a Master’s Group Project (ESM 401 series) or Eco-E Project (ESM 402 series) equivalent to 1 IWC. Faculty advisors are expected to meet with master’s groups weekly during the academic year.

Course Equivalencies:
\begin{itemize}
    \item ESM 401A/ESM 402A: 4 units = 0.29 IWC 
    \item ESM 401B/ESM 402B: 4 units = 0.29 IWC
    \item ESM 401C/ESM 402C: 4 units = 0.29 IWC
    \item ESM 401D/ESM 402D: 2 units = 0.13 IWC
\end{itemize}

Every full-time faculty member will teach and/or co-teach core courses, elective courses and seminars in the MESM and PhD programs equivalent to at least (a) 2.5 IWC in addition to advising a project or (b) 3.5 IWC if not advising a project. 

Course Equivalencies:
\begin{itemize}
    \item MESM core and elective courses: 4 units = 1 IWC; 2 units = 0.5 IWC.
    \item Co-taught 4-unit MESM core course = 0.6 IWC per instructor.
    \item Co-taught 4-unit MESM elective course = 0.5 IWC per instructor.
    \item MESM lab course: 4 or 5 units = 1 IWC.
    \item Teaching an additional section (e.g., ESM 263) = 0.67 IWC.
    \item PhD core courses: ESM 510 (1 unit) = 0.25 IWC; ESM 512 (2 units) = 0.5 IWC; ESM 514 (4 units) = co-taught at 1 IWC per instructor.
    \item PhD seminar course: 2 units = 0.5 IWC; 4 units = 1 IWC.
\end{itemize}

\item The ERI directorship is for a 5-year term with annual reappointment.  The appointment comes with a 50\% teaching load reduction. Therefore, my four-year annual teaching load in Geography is 1-0-1-1 (an average of 0.75 IWC per year), and in Bren it is advising one group project per year, and co-teaching the MESM core course ESM 203 starting in Fall 2018 (a total of 1.6 IWC per year.

}
\end{enumerate}

%\newpage

\vspace{0.5cm}
{\bf Catalog Courses:}
%\vspace{0.15cm}
% UC Bio-bib Catalog Courses Table
% Created on 2020-10-05 14:54

\begin{longtable}{lp{6.5cm}p{1cm}rrrp{2cm}}
Qtr & Course & Class Type & Units & Hrs/Wk & Enrollment & ESCI/Written Evals Avail.\\
\hline 
\endhead 
F16 & Individual Study - PhD Exam & Tut &  & 2 & 1.0 & No/No \\ 
W17 & Individual Study - PhD Exam & Tut &  & 2 & 1.0 & No/No \\ 
S17 & Biogeography & Lec & 4.0 & 3 & 12.0 & Yes/Yes \\ 
S17 & Biogoegraphy & Lec & 4.0 & 3 & 17.0 & Yes/Yes \\ 
S17 & Group Project - A & Lec & 4.0 & 1 & 4.0 & No/No \\ 
S17 & Individual Study - PhD Exam & Tut &  & 2 & 2.0 & No/No \\ 
S17 & Intro to Geographical Research & Tut & 2.0 & 1 & 1.0 & No/No \\ 
F17 & Group Project - B & Dis & 4.0 & 1 & 4.0 & No/No \\ 
F17 & Individual Study - PhD Exam & Tut &  & 2 & 2.0 & No/No \\ 
W18 & Group Project - C & Dis & 4.0 & 1 & 4.0 & No/No \\ 
W18 & Individual Study - PhD Exam & Tut &  & 2 & 1.0 & No/No \\ 
W18 & PhD Dissertation & Tut &  & 2 & 1.0 & No/No \\ 
S18 & Group Project - A & Lec & 4.0 & 1 & 5.0 & No/No \\ 
S18 & Group Project - D & Dis & 2.0 & 1 & 4.0 & No/No \\ 
S18 & Individual Study - PhD Exam & Tut &  & 2 & 1.0 & No/No \\ 
S18 & Intro to Geographical Research & Tut & 2.0 & 1 & 1.0 & No/No \\ 
S18 & PhD Dissertation & Tut &  & 2 & 1.0 & No/No \\ 
F18 & Earth System Science & Lec & 4.0 & 1 & nan & Yes/Yes \\ 
S19 & Water, Energy, and Ecosystems & Lec & 4.0 & 1 & nan & Yes/Yes \\ 
 
\end{longtable}



\vspace{0.5cm}
{\bf MESM Projects Advised}
%\vspace{0.15cm}
% UC Bio-bib MESM Projects Table
% Created on 2020-10-04 17:20

\begin{longtable}{p{1cm}p{2.5cm}p{3cm}p{2cm}p{2cm}p{2cm}p{2cm}}
Year & Project Title & Students & Q3 & Q4 & Q5 & Q7\\
\hline 
\endhead 
2019 - 2020 & Green infrastructure for coral reef protection in Hawaii' &  &  &  &  &  \\ 
2018 - 2019 & Determining conservation research opportunities at the TNC Point Conception Dangermond Preserve & Brad Anderson,  &  &  &  &  \\ 
2017 - 2018 & Determining conservation research opportunities at the TNC Point Conception Dangermond Preserve & Sravan Chalasani , Karina Herrera, John Sisser, Zach Voss & 67\% Strongly Agree, 33\% Somewhat Agree & 67\% Excellent, 33\% Very Good & 67\% Always, 33\% Usually & 66\% Excellent, 33\% Satisfactory \\ 
\end{longtable}



\vspace{1cm}
{\bf Undergraduate Projects Directed:}
{\setlength{\extrarowheight}{3.5pt}
\tablehead{}
\begin{supertabular}{lp{11cm}cc} 
Student & Project & Chair/ & Yr Deg \\
 & & Member & Compl. \\
 \hline
%Madison Gray & History of Urban Design in Edinburgh & Chair &  2008 \\
%Eric Wilder & An assessment of environmental planning in Humboldt County, CA & Chair & 2010 \\ 
\end{supertabular}
}
None.

\vspace{1cm}
{\bf Graduate Degree Committees, MA/MS Committees:}
\vspace{0.25cm}
\input{tex/GraduateAdvisingMS}
None.


\vspace{0.5cm}
{\bf Graduate Degree Committees, Ph.D. Committees:}
\vspace{0.25cm}
\input{tex/GraduateAdvisingPhD}


\vspace{0.5cm}
{\bf Postdoctoral Scholars Supervised:}
%\vspace{0.25cm}
% UC Bio-bib Catalog Postdoctoral Advising Table
% Created on 2020-10-04 17:20

\begin{longtable}{lp{1.5cm} p{3.5cm}p{4.5cm}}
Postdoctoral Researcher & Years & Affiliation & Current Employment\\
\hline 
\endhead 
Marc Mayes & 2016 -  & UCSB, Geography &  -   \\
Farai Kaseke & 2018 -  & UCSB, Geography &  -   \\
\end{longtable}




\vspace{0.5cm}
{\bf Other Teaching Contributions:}

None

%{\setlength{\extrarowheight}{3.5pt}
%\tablehead{}
%\begin{supertabular}{cp{16cm}} 
%- & Coordinator and instructor for the UCSB Spatial Perspectives on Curriculum Enhancement (SPACE) workshop, July 2007 \\
%- & Took the Geography 20 honors section on a field trip to UC San Diego to attend the Surfing, Arts, Sciences, and Issues Conference at Scripps Institute, February 2007 \\
%- & Mentor/advisor for Sam Boysel (San Marcos High School student) for his senior honors science project. \\
%- & Faculty undergraduate advisor, 2008-09, Geography \\
%\end{supertabular}
%}

\vspace{0.5cm}
\textbf{PART III.  PROFESSIONAL ACTIVITIES}

\vspace{0.5cm}
\textbf{Lectures and Seminars Presented:}
\vspace{0.25cm}
{\setlength{\extrarowheight}{3.5pt}
% UC Bio-bib Lectures Table
% Created on 2020-10-04 22:12

\begin{longtable}{lp{10.0cm}p{4.5cm}}
Month/Year & Title & Meeting/Place\\
\hline 
\endhead 
8/2018 & Session on ``Vegetation dynamics and ecosystem resilience under global climate change'', Ecological Society of America Annual Meeting, New Orleans, LA \\
8/2018 & Coupled Natural-Human Systems Short Course, University of California, Irvine \\
8/2019 & 44th New Phytologist Symposium, Accra, Ghana \\
\end{longtable}


}


\vspace{0.5cm}
\textbf{Conference Posters and Presentations:}
\vspace{0.25cm}
{\setlength{\extrarowheight}{3.5pt}
% UC Bio-bib Proceedings Table
% Created on 2020-10-04 17:16

\begin{longtable}{lp{10.0cm}p{4.5cm}}
Month/Year & Title & Meeting/Place\\
\hline 
\endhead
 
10/2018 & {\bf Regional migration, global climate change,
and the future of irrigation on Kenyan farms}. Lopus, S. Waldman, K., Guido, Z., Caylor, K., Evans, T.  (Invited)  & International Conference on Migration, Environment and Climate: What risk inequalities?  \\
  
12/2018 & {\bf Fusing Remote Sensing and Demography to Assess which Sub-Saharan African Cities Are Growing the Fastest}. Tuholske, C., Caylor, K.K.  (Invited)  & American Geophysical Union Fall Meeting \\
  
12/2018 & {\bf A Convolutional Neural Network Approach to Segmenting Smallholder Agriculture}. Ryan Barry Avery, Kelly K Caylor, Lyndon D Estes, Ronald Eastman, Su Ye, Lei Song, Kaixi Zhang, Sitian Xiong, Dennis McRitchie and Tammy Woodard  (Invited)  & American Geophysical Union Fall Meeting \\
  
12/2018 & {\bf Quantifying the Effect of Farmer Management Decisions on Maize Yield in Zambia}. Michael Cecil, Katherine Baylis, Jordan Blekking, Kelly K Caylor, Tom P Evans, Megan Konar, Justin Sheffield, Noemi Vergopolan, Kurt Waldman, Yi Zhao, and Lyndon D Estes  (Invited)  & American Geophysical Union Fall Meeting \\
  
12/2018 & {\bf Using active learning to quantify how training data errors impact classification accuracy over smallholder-dominated agricultural systems}. Lyndon D Estes, Stephanie R Debats, Dennis McRitchie, Ronald Eastman, Lei Song, Tammy Woodard, Sitian Xiong, Su Ye, Kaixi Zhang, Ryan Barry Avery, and Kelly K Caylor  (Invited)  & American Geophysical Union Fall Meeting \\
  
12/2018 & {\bf Reimagining high-resolution ecosystem monitoring with low-cost autonomous sensing}. Kelly K Caylor, Elizabeth Forbes, Grace Lewin, Mark Hirsh  (Invited)  & American Geophysical Union Fall Meeting \\
  
12/2018 & {\bf Seasonal and diurnal drone and ground-based thermal, multispectral and hyperspectral imaging to quantify responses of California oak woodland productivity and evapotranspiration to extreme climate conditions}. Marc Mayes, Kelly Caylor  (Invited)  & American Geophysical Union Fall Meeting \\
  
4/2019 & {\bf Integrating humans and machines to map smallholder-dominated agricultural frontiers}. Estes, Lyndon Despard; Song, Lei; Ye, Su; Avery, Ryan; McRitchie, Dennis; Debats, Stephanie; Xiong, Sitian; Eastman, Ron; Woodard, Tammy; Caylor, Kelly  (Invited)  & 4th Open Science Meeting of the Global Land Programme \\
  
4/2019 & {\bf }.   (Invited)  &  \\
 \end{longtable}


}

%\newpage
\vspace{0.5cm}
\textbf{Grants and Contracts:}
\vspace{0.25cm}
{\setlength{\extrarowheight}{3.5pt}
% UC Bio-bib Funding Table
% Created on 2020-10-11 20:58

\begin{longtable}{p{1.75cm}>{\raggedright}p{2.75cm}p{5.5cm}p{1cm}p{1.25cm}p{1.25cm}p{1cm}}
Year & Source & Title & Role & Amount & Personal Share & New/Cont.\\
\hline 
\endhead 
\end{longtable}


}

\vspace{0.5cm}
\textbf{Reviewing and Refereeing Activity:}
\vspace{0.2cm}
% UC Bio-bib Reveiwer Activity Table
% Created on 2020-10-04 17:16

\begin{longtable}{llp{12cm}}
Year & Activity & Journal/Agency\\
\hline 
\endhead 
2018 & Promotion case reviewer & SUNY-ESF  \\
2018 & Promotion case reviewer & University of Washington  \\
2018 & Referee & Agricultural and Forest Meteorology  \\
2018 & Referee & Biology Letters  \\
2018 & Referee & Communications Biology  \\
2018 & Referee & Journal of Water Resources Planning \& Management  \\
2018 & Referee & Landscape Ecology  \\
2018 & Referee & Nature Climate Change  (3)  \\
2018 & Referee & Nature Communications  \\
2018 & Referee & Progress in Physical Geography  \\
2018 & Referee & Water Resources Research  \\
2018 & Tenure case reviewer & University of Toronto  \\
2019 & Referee & Water Resources Research  \\
\end{longtable}



\vspace{0.25cm}
\textbf{Special Appointments:}

%\hspace{0.5cm} 2004-2008, Editorial Board, \textit{Geographical Analysis}
\hspace{0.5cm} 2016-present, Editorial Board, \textit{Environmental Research Letters, Reviews}

\vspace{0.25cm}
\textbf{Other Professional Contributions:}
\hspace{0.5cm}
\begin{tabular}{lp{2cm}p{12cm}}
% 2011-present &  Volunteer Scientist &New York City Department of Public Health.  Assisting their staff in developing methodology to improve their use of FITNESSGRAM data for monitoring, screening, and policy evaluation of childhood obesity in NYC public schools. \\
% 2015-16 & Consultant & (\$16,464) \emph{Quantile regression of longitudinal BMI growth trajectories.}, New York City Department of Health and Mental Hygiene \\
%  2015 & Participant & NSF Future Directions for Mathematics, Measurement, and Statistics Program
\end{tabular}

%\newpage

\vspace{0.5cm}
\textbf{PART IV.  SERVICE}

\vspace{0.5cm}
\textbf{University Service:}
\vspace{0.2cm}
% UC Bio-bib Professional Generic Service Table
% Created on 2020-10-05 14:54

\begin{longtable}{llp{12cm}}
Year & Role & Service\\
\hline 
\endhead 
2018 - 2019 & Chair & Earth Research Institute, Personnel Committee \\
2018 - 2019 & Director & Earth Research Institute, University of California, Santa Barbara \\
2018 - 2019 & Ex-Officio Member & Earth Research Institute Advisory Committee, UCSB \\
2018 - 2019 & Member & Campus Data Science Initiative Working Group \\
2018 - 2019 & Member & North Campus Open Space Administrative Advisory Group \\
2018 - 2019 & Member & Data Science Curriculum Committee \\
2018 - 2019 & Participating Faculty & Interdepartmental PhD Emphasis in Environment and Society \\
2019 - 2020 & Chair & Earth Research Institute, Personnel Committee \\
2019 - 2020 & Ex-Officio Member & Earth Research Institute Advisory Committee, UCSB \\
2019 - 2020 & Member & Dangermond Endowed Chair Search Committee, Office of Research \\
2019 - 2020 & Member & North Campus Open Space Administrative Advisory Group \\
2019 - 2020 & Member & Campus Data Science Initiative Working Group \\
2019 - 2020 & Member & Data Science Curriculum Committee \\
2019 - 2020 & Participating Faculty & Interdepartmental PhD Emphasis in Environment and Society \\
\end{longtable}



\vspace{0.5cm}
\textbf{Department Service:}
\vspace{0.2cm}
% UC Bio-bib Professional Generic Service Table
% Created on 2020-10-05 14:51

\begin{longtable}{llp{12cm}}
Year & Role & Service\\
\hline 
\endhead 
2018 - 2019 & Chair & Space Committee, Department of Geography, University of California, Santa Barbara \\
2018 - 2019 & Member & Graduate Admission Committee, Department of Geography, University of California, Santa Barbara \\
2018 - 2019 & Member & Vegetation Data Science Faculty Search Committee, Department of Geography, UCSB \\
2019 - 2020 & Chair & Space Committee, Department of Geography, University of California, Santa Barbara \\
2019 - 2020 & Member & Group Project Committee, Bren School of Environmental Science and Management, University of California, Santa Barbara \\
2019 - 2020 & Member & Dangermond Endowed Chair Search Committee, Department of Geography \\
2019 - 2020 & Member & Graduate Admission Committee, Department of Geography, University of California, Santa Barbara \\
\end{longtable}



\vspace{0.5cm}
\textbf{Public Service:}
\vspace{0.2cm}
% UC Bio-bib Professional Generic Service Table
% Created on 2020-10-05 19:27

\begin{longtable}{llp{12cm}}
Year & Role & Service\\
\hline 
\endhead 
2018 & Co-Convenor & AGU Fall Meeting, Session on “Indicators of Plant Water Availability and Stress in Drought-Prone Forests at a Range of Spatial and Temporal Scales” \\
\end{longtable}





\end{document}





